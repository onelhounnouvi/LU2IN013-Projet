\documentclass[a4paper]{article}
\usepackage{adjustbox}

\title{Automatisation de la cryptanalyse des cryptosystèmes classiques
à l’aide d’algorithmes modernes}
\author{Helder Brito\\O'nel Hounnouvi}
\date{}

\begin{document}
\maketitle 
\section{Substitution monoalphabétique}
\subsection{Introduction}
La substitution monoalphabétique est des plus anciennes méthodes de chiffrement. Elle consiste
à remplacer dans le message clair une lettre donnée de l'alphabet par une autre lettre. Voici un exemple:

\vspace{1em}
    \begin{adjustbox}{width=\textwidth,center}
        \begin{tabular}{|c|c|c|c|c|c|c|c|c|c|c|c|c|c|c|c|c|c|c|c|c|c|c|c|c|c|}
            \hline
            A & B & C & D & E & F & G & H & I & J & K & L & M & N & O & P & Q & R & S & T & U & V & W & X & Y & Z \\
            \hline
            X & Y & Z & A & B & C & D & E & F & G & H & I & J & K & L & M & N & O & P & Q & R & S & T & U & V & W \\
            \hline
        \end{tabular}
    \end{adjustbox}
\vspace{1em}

Le message \textit{SUBSTITUTION} devient \textit{PRYPQFQRQFLK}. \\

L'alphabet latin comporte 26 lettres. Cela permet donc de construire $26! = 4 \times 10^{26}$ permutations. Soit de l'ordre de $2^{88}$.
Sachant qu'environ $2^{58}$ secondes se sont écoulées depuis la création de l'univers, il serait impossible d'explorer toutes les permutations.
Ce chiffre donne une impression de sûreté qui est toutefois trompeuse\ldots

\subsection{Cryptanalyse}
La substitution monoalphabétique possède de grosses faiblesses structurelles. Les chiffres utilisant cette méthode sont faciles à casser par 
analyse fréquentielle. Par exemple, dans un texte français, il y a toujours plus de E que de W.
\end{document}